% Metódy inžinierskej práce

\documentclass[10pt,twoside,slovak,a4paper]{coursepaper}

\usepackage[slovak]{babel}
%\usepackage[T1]{fontenc}
\usepackage[IL2]{fontenc} % lepšia sadzba písmena Ľ než v T1
\usepackage[utf8]{inputenc}
\usepackage{graphicx}
\usepackage{url} % príkaz \url na formátovanie URL
\usepackage{hyperref} % odkazy v texte budú aktívne (pri niektorých triedach dokumentov spôsobuje posun textu)

\usepackage{cite}
%\usepackage{times}

\pagestyle{headings}

\title{Modelovanie softvéru autonómnych aut\thanks{Semestrálny projekt v predmete Metódy inžinierskej práce, ak. rok 2021/22, vedenie: Vladimír Mlynarovič}} % meno a priezvisko vyučujúceho na cvičeniach

\author{Peter Bartoš\\[2pt]
	{\small Slovenská technická univerzita v Bratislave}\\
	{\small Fakulta informatiky a informačných technológií}\\
	{\small \texttt{xbartosp2@stuba.sk}}
	}

\date{\small 12. október 2021} % upravte



\begin{document}

\maketitle

\begin{abstract}
Prvý príchod aut sa datuje už ďaleko do minulosti a všeobecný fakt je, že nám umožnil zdolávať pomerne dlhé vzdialenosti za oveľa kratší čas. 
Táto zvýšená rýchlosť nám skrátila čas na spracovávanie informácii a tým zväčšila šancu ohroziť a poškodiť naše okolie počas riadenia tohto vozidla. 
Tiež vieme, že človek nedokáže konkurovať počítaču pri rýchlosti spracúvania informácii. Tak prečo nie autonómne autá? Funkcie auta by boli ovládané počítačom alebo umelou inteligenciou, 
ktoré by rozhodovali kedy pridať plyn alebo brzdiť, kedy sa preradiť do iného pruhu alebo kedy zastaviť na prechode pre chodcov. 
Prototypy týchto aut už v dnešnej dobe existujú a tento článok analyzuje ich modelovanie v oblasti softvérového inžinierstva.
\end{abstract}



\section{Úvod}

%Motivujte čitateľa a vysvetlite, o čom píšete. Úvod sa väčšinou nedelí na časti.

%Uveďte explicitne štruktúru článku. Tu je nejaký príklad.
%Základný problém, ktorý bol naznačený v úvode, je podrobnejšie vysvetlený v časti~\ref{nejaka}.
%Dôležité súvislosti sú uvedené v častiach~\ref{dolezita} a~\ref{dolezitejsia}.
%Záverečné poznámky prináša časť~\ref{zaver}.

Niet pochýb, že autonómne riadená doprava by bola bezpečnejšia ako tá dnešná.
Štúdia NHTSA(ref), ktorá sa zaoberá hlavnými dôvodmi dopravných nehôd, tvrdí, 
že 2\% sú zapríčinené prostredím, d'alšie 2\% sú spôsobené nedostatkami vozidiel, 
neznáme príčiny tvoria taktiež 2\% a celých 94\% je tvorených našou (l'udskou) chybou.


\section{Pokračovanie}

Gunči, linči.

Interesting.
Hmmmm, yeeees.~\ref{nejaka}.
Coooly pooly.~\ref{dolezita} a~\ref{dolezitejsia}.
Likey bikey.~\ref{zaver}.



\section{Nejaká časť} \label{nejaka}

Z obr.~\ref{f:rozhod} je všetko jasné. 

\begin{figure*}[tbh]
\centering
%\includegraphics[scale=1.0]{diagram.pdf}
Aj text môže byť prezentovaný ako obrázok. Stane sa z neho označný plávajúci objekt. Po vytvorení diagramu zrušte znak \texttt{\%} pred príkazom \verb|\includegraphics| označte tento riadok ako komentár (tiež pomocou znaku \texttt{\%}).
\caption{Rozhodujúci argument.}
\label{f:rozhod}
\end{figure*}



\section{Iná časť} \label{ina}

Základným problémom je teda\ldots{} Najprv sa pozrieme na nejaké vysvetlenie (časť~\ref{ina:nejake}), a potom na ešte nejaké (časť~\ref{ina:nejake}).\footnote{Niekedy môžete potrebovať aj poznámku pod čiarou.}

Môže sa zdať, že problém vlastne nejestvuje\cite{Coplien:MPD}, ale bolo dokázané, že to tak nie je~\cite{Czarnecki:Staged, Czarnecki:Progress}. Napriek tomu, aj dnes na webe narazíme na všelijaké pochybné názory\cite{PLP-Framework}. Dôležité veci možno \emph{zdôrazniť kurzívou}.


\subsection{Nejaké vysvetlenie} \label{ina:nejake}

Niekedy treba uviesť zoznam:

\begin{itemize}
\item jedna vec
\item druhá vec
	\begin{itemize}
	\item x
	\item y
	\end{itemize}
\end{itemize}

Ten istý zoznam, len číslovaný:

\begin{enumerate}
\item jedna vec
\item druhá vec
	\begin{enumerate}
	\item x
	\item y
	\end{enumerate}
\end{enumerate}


\subsection{Ešte nejaké vysvetlenie} \label{ina:este}

\paragraph{Veľmi dôležitá poznámka.}
Niekedy je potrebné nadpisom označiť odsek. Text pokračuje hneď za nadpisom.



\section{Dôležitá časť} \label{dolezita}




\section{Ešte dôležitejšia časť} \label{dolezitejsia}




\section{Záver} \label{zaver} % prípadne iný variant názvu



%\acknowledgement{Ak niekomu chcete poďakovať\ldots}


% týmto sa generuje zoznam literatúry z obsahu súboru literatura.bib podľa toho, na čo sa v článku odkazujete
\bibliography{literatura}
\bibliographystyle{abbrv} % prípadne alpha, abbrv alebo hociktorý iný
\end{document}
